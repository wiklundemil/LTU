\documentclass[a4paper,12pt]{article}
\usepackage[utf8]{inputenc}  % Denna rad är datorspecifik
\usepackage[swedish]{babel}

\title{Om \LaTeX\ och prov 1 i momentet \\ \emph{D. Texter och \LaTeX}
  i kursen D0015E} 

\author{H{\aa}kan Jonsson\\ ~\\ Institutionen för system- och
  rymdteknik \\ Luleå tekniska universitet}
\begin{document}

\maketitle

\begin{abstract}
  Prov 1 går ut på att skapa ett dokument identiskt med ett annat med
  hjälp av \LaTeX. Här ges en förklaring till innehållet i den startfil
  du får. 
\end{abstract}

\section{Allmänt}

\LaTeX\ erbjuder fantastiska möjligheter att producera professionellt
typsatt text, men också lika stora möjligheter att göra fel. Du måste
därför vara \emph{mycket noggrann} då du skriver ditt
\LaTeX-manus. Minsta tecken har betydelse och även små fel, som att
råka använda en stor bokstav där det ska vara en liten, gör att varje
försök att generera ett dokument från manuset rasar samman som ett
korthus. 
\begin{center}
  \noindent
  \fbox {
    \parbox{0.8\textwidth}{
      Skriv manuset i mycket små steg och generera dokument mellan
      stegen, så får du eventuella fel i småportioner och kan enklare
      åtgärda dem. Jag gör alltid så här och jag råder även dig att
      göra det. 
    }
  }
\end{center}

\subsection{Layout}

Skilj mellan \emph{innehåll} och \emph{utseende}. Koncentrera dig på
innehållet, dvs att text och formler får korrekt innehåll, och överlåt
sen åt \LaTeX\ att bestämma utseendet. Kanske detta känns ovant, för
WYSIWYG\footnote{What-You-See-Is-What-You-Get.}-program som t ex
Microsoft~Word bygger ju på att skribenter också bestämmer
layouten. Men \LaTeX\ är inte något WYSIWYG-program. \LaTeX\ är
programmerat med hjälp av de bästa layoutreglerna och layoutar bättre
än oss alla i 99 fall av 100. Om du trots det tycker att layouten inte
är bra så vänta med uttryckliga ändringar av layouten tills du fått
allt innehåll korrekt(!) Annars skapar du bara en massa problem för
dig själv. Även små ändringar av innehållet kan orsaka mycket stora
ändringar av hur \LaTeX\ utformar dokumentets layout. 

\subsection{Kommentarer}

\LaTeX-systemet ignorerar allt på en rad efter ett procenttecken
(\verb!%!). Med \verb!%! kan man således inkludera kommentar i sitt
manus som inte påverkar det färdiga dokumentet men som kan vara
viktiga för någon som läser manuset. Om du kikar på filen
\texttt{koch-first.tex}, ett inledande men ofullständigt manus, ser du
många sådana kommentarer som inkluderats som en hjälp till dig. Var
och en av dem listar nämligen något som ska vara med i ditt färdiga
manus och ungefär på den plats kommentaren förekommer.  

\subsection{Kommandon och omgivningar}

Ett \LaTeX-kommando startar med ett bakåt-snedstreck (\verb!\!). Sen
följer kommandots namn och eventuella argument\footnote{Jämför med hur
  funktioner och metoder används i \texttt{python}.}. Vill man t ex ha
\textbf{text med fet stil} använder man kommandot \label{sec:textbf}
\verb!\textbf{text med fet stil}!. Vill man istället att \emph{text
  ska framhävas} så skriver man \verb!\emph{text ska framhävas}!. 
\label{sec:emph} Det finns väldigt många olika kommandon och ett råd
är att snarare googla efter/slå upp detaljer då du behöver dem än att
lära dig dem alla utan till. 

Förutom kommandon finns det även \LaTeX-omgivningar
(\textit{environments} på engelska). Dessa inleds med \verb!\begin!,
  avslutas med \verb!\end! och påverkar det som skriv mellan
\texttt{begin-end}. Exempel: Om man vill 
\begin{center}
  centrera text
\end{center}
använder man omgivningen \texttt{center} och skriver
\begin{verbatim}
\begin{center}
  centrera text
\end{center}
\end{verbatim}
Om du läser manuset till detta dokument kommer du att se att jag
använt omgivningen \texttt{verbatim} för att får in
centreringsomgivningen ovan i min text.  

\subsection{Tomrum}

En tomrad bland text markerar nytt stycke. Detta kan märkas genom att
första raden dras in eller att \LaTeX\ lägger in lite tomrum vertikalt
mellan styckena. Var noga med eventuella tomrader du väljer att ta med
i ditt manus. Däremot spelar godtyckligt många mellanslag mellan ord i
text ingen annan roll än att de byts ut mot ett lagom stort mellanrum
oavsett hur många de är. Därför kan man i regel fritt \emph{indentera
  rader}, dvs stoppa in mellanslag i början av rader för att visa på 
hur på varandra följande rader hör ihop, och därigenom dramatiskt öka
manusets läsbarhet. För din egen skull - gör det!
\begin{center}
  \noindent
  \fbox {
    \parbox{0.8\textwidth}{
      Jag brukar indentera sånt som är inne i en omgivning minst 2
      mellanslag relativt närmast omgivande omgivning. Det gör att jag
      enklare ser vad som tillhör vad. Om jag inte hinner indentera
      när jag skriver så återvänder jag till texten och gör det
      senare. Jag råder dig att indentera på samma sätt som mig, och
      även då du programmerar datorer. I vissa språk, t ex
      \texttt{python}, har indenteringen även betydelse för
      slutresultatet.  
    }
  }
\end{center}

\section{Om prov 1}

Här följer en genomgång av innehållet i \texttt{koch-first.tex} med 
kommentarer. 

\subsection{Inledningen}

Varje manus inleds med lite allmänna deklarationer och importer. 

\begin{itemize}
  \item \verb|\documentclass[12pt,a4paper]{article}| säger att detta
    är ett manus för en artikel som ska typsättas med 12 punkters 
    typsnitt och passa på ett A4-ark. 
  \item \verb|\usepackage[utf8]{inputenc}| anger den teckenkodning som
    används. Detta är datorspecifikt så du kan behöva ändra
    \texttt{uft8} till något annat t ex \texttt{latin1}.  

    \verb|\usepackage| är ett kommando som importerar andra kommandon
    och omgivningar än de som det grundläggande \LaTeX-systemet består
    av. Kommandot gör det importerade tillgängligt när man skriver
    manuset. 
  \item \verb|\usepackage{graphicx}| gör det möjligt att inkludera
    grafik, t ex JPEG-bilder eller PDF-filer. Detta gör man i
    praktiken sen med kommandot \verb|\includegraphics|. 
  \item \verb|\usepackage{amsmath, amsthm, amssymb}| utökar stödet för
    matematik. 
  \item \verb|\newtheorem{theorem}{Theorem}| introducerar
    \texttt{theorem}, en ny omgivning för matematiska satser/teorem.  
\end{itemize}

Inledningen kan även spalta upp titel, författarnamn, datum,
rubrikstilar och annat. Så görs i manuset för det dokument du just nu
läser men inte i \texttt{koch-first.tex}.  

\subsection{Själva dokumentet}

Dokumentdelen av manuset består av omgivningen \texttt{document}. Allt
som skrivs inom  
\begin{verbatim}
\begin{document}

\end{document}
\end{verbatim}
hamnar i det färdiga dokumentet. Text som skrivs in hamnar i
dokumentet i den ordning den skrivs (uppifrån och ned, vänster till
höger). Kommandon och omgivningar säger om text är något speciellt som
t ex rubriker, fotnoter, matematiska formler mm. \LaTeX\ väljer ett
sätt att typsätta på, och bestämma layouten på, som passar texten och
den pappersstorlek man angett i inledningen. 

\begin{itemize}
  \item \verb|\section| säger att här ska det in en
    avsnittsrubrik. Argumentet till kommandot, i detta fall
    \texttt{Headline...}, är rubriktexten. \LaTeX\ kommer att välja en
    lagom storlek på rubriken (och övrig text) som passar till det
    slutgiltiga dokumentets storlek.  
  \item \verb|\emph| har vi gått igenom ovan på sidan
    \pageref{sec:emph}. 
  \item \verb|\cite| används för att referera till en referens, som
    beskrivs i omgivningen \texttt{thebibliography} i slutet av
    manuset. En referens är typiskt en artikel eller rapport. 
    Här refererar vi referensen \texttt{koch}, vilket är
    varför vi skriver\verb|\cite{koch}|. \LaTeX\ kommer att byta ut
    \verb|\cite{koch}| mot ett löpnummer inom hakparenteser. 

    Har man flera referenser ger
    \LaTeX\ varje ett eget nummer och ser till att utbytena blir
    rätt. Detta är mycket smidigt eftersom man inte själv behöver
    hålla reda på löpnumren, utan kan helt överlåta detta åt
    \LaTeX-systemet. Däremot måste man (förstås) själv definiera
    referenserna och ge dem unika nyckelord (i detta fall
    \texttt{koch}). 

  \item \verb|figure| är en omgivning för figurer som man vill ha
    bildtext till, t ex bilder. \verb|[h]| anger att vi vill ha
    figuren just \textbf{h}är, och inte någon annanstans(!) Beroende
    på \LaTeX:s layoutregler \emph{kan} figuren ändå hamna på annan
    plats, vilket man då kan åtgärda. Men sådan åtgärd ska man alltså
    vidta först när allt annat innehåll är korrekt (inte omedelbart
    när man till sin fasa ser att figuren hamnar på fel ställe).  

  \item \verb|\label| är en etikett för något i dokumentet som man sen
    vill kunna referera till som avsnitt, ekvationer, sidor, bilder,
    tabeller mm. Till \verb|\label| fogar man ett unikt namn på
    etiketten som man sen känner igen. En etikett för ett inledande
    avsnitt skulle kunna vara \verb|sec:intro| medan man för en
    ekvation kanske använder \verb|eq:pythagoras|. De här prefixen
    \texttt{sec:} och \texttt{eq:} är inget \LaTeX-systemet kräver,
    utan bara till för att man snabbt ska se vad för något en referens
    leder till. För en figur brukar jag skriva t ex
    \verb|\label{fig:circle}|, där \texttt{fig:} gör det lätt att
    skilja etiketten från etiketter för avsnitt och ekvationer. Du får
    döpa dina etiketter som du vill men håll dig till vanliga
    bokstäver i etikettnamnen. 

    Avsnitt, ekvationer, figurer och annat numreras separat inom
    respektive sort och stigande. Det går dock att ändra detta (men
    det behöver man sällan). 

  \item \verb|\centering| inne i en \texttt{figure} ger centrering av
    figuren. Omgivningen \texttt{center} fungerar likartat men ska
    inte användas just här. 

  \item Kommandot \verb|\includegraphics| säger att här ska grafik
    in från en fil vars namn ett argument anger. Vår fil heter
    \texttt{snowflake.jpg} och dessutom föreskriver vi att vi vill att
    bilden får en bredd på 9 cm. Detta behövs egentligen inte, men då 
    får bilden sin naturliga storlek som kan vara direkt
    olämplig. 

    Bilder är något \LaTeX\ inte klarar av särskilt väl. Bildhantering
    i \LaTeX\ är krångligt, blir ofta fel, och därför har jag
    inkluderat hela det kommando du ska använda för att få in bilden 
    på snöflingans första konstruktionssteg. 

  \item \verb|\caption| anger en bildtext.

  \item Omgivningen \texttt{quote} lägger in ett längre citat, och
    indenterar från både höger och vänster.

  \item \verb|\textit| ger kursiv stil. 

  \item Kommandot \verb|\ref| används för att referera till något i
    dokumentet som märkts ut med en etikett. Om man t ex lagt in en
    (1) figur i sitt dokument, och i denna inkluderat etiketten
    \verb|\label{fig:snowflakes}|, så kommer
    \verb|\ref{fig:snowflakes}| i texten sen att bytas ut mot en
    1:a. Skulle figuren vara den andra i ordning, sker utbytet
    istället till en 2:a. 

  \item \verb|\subsection| fungerar som \verb|\section| men används
    för rubriknivån omedelbart under den översta. I \LaTeX\ finns även
    ytterligare en nivå, som man får med \verb|\subsubsection|. 

  \item Omgivningen \texttt{theorem} typsätter en typisk matematisk
    sats. Fyll i med påståendet i satsen mellan \texttt{begin} och
    \texttt{end}.  

  \item \verb|proof| används för bevis. I ditt manus ska du skriva kod
    för två bevis, som alltså läggs in i omgivningen
    \verb|proof|. Notera att \texttt{proof} automatiskt lägger till en
    passande slutsymbol.  

  \item Matematisk löptext märks i \LaTeX\ ut genom att omges av
    dollartecken (\verb|$|). Skriver man t ex
    \verb|$f(x)+f(2x)/2+f(3x)/3$| typsätts det som matematisk text,
    dvs 
    \begin{displaymath}
       f(x)+f(2x)/2+f(3x)/3,
    \end{displaymath}
    och inte som vanlig text, dvs f(x)+f(2x)/2+f(3x)/3. Lägg noga
    märke till stilskillnaden. Det är lika fel att typsätta
    matematiska formler som vanlig löptext som att säga att löptext
    ska se ut som matematisk text. I prov~1 finns flera tillfällen när
    det gäller att välja rätt typ av text. 

    (Matematiska formler kan även märkas ut på andra sätt.)

  \item \verb|\Delta| ger den grekiska bokstaven stora D~(dvs
    $\Delta$). 

  \item \verb|N_i| ger i matematisk text $N_i$, dvs
    understrykningstecken ger (i matematiska formler) ett index. 

  \item \verb|displaymath| ger en formel som typsätts fritt i ett
    större vertikalt mellanrum. Man behöver inga dollartecken i
    omgivningen, utan \LaTeX\ förutsätter att det som skrivs där måste
    vara matematik.  

    Samma matematiska text kan i regel antingen typsättas i löptext,
    med dollartecken, eller fritt med bl a \texttt{displaymath}; se
    även \texttt{equation} nedan.  

  \item \verb|cases| används för ekvationer med många fall. 

  \item Skriver man \verb|\text{en mening}| i matematiska formler
    typsätts \texttt{en mening} ändå som vanlig löptext. Detta är
    användbart när man vill lägga till förklaringar.  

  \item \verb|N_{n-1}| ger ett mera komplicerat index på
    $N$. Klammerparenteser kan alltid användas för att gruppera
    ihop. Skriver man t ex \label{sec:maas} 
    \begin{center}
      \verb|$B_{N_{i+1}+M_{j+2i}}$|
    \end{center}
    får man 
    \begin{displaymath}
      B_{N_{i+1}+M_{j+2i}}.
    \end{displaymath}
    På samma sätt kan t ex omfattande exponenter och integralgränser
    skapas.  

  \item \verb|equation| används som \texttt{displaymath} men ger
    automatiskt ett ekvationsnummer som skrivs ut i marginalen. Koden  
    \begin{verbatim}
\begin{equation}
  \label{eq:first}
  f(x) = x^2 + 2x + 9
\end{equation}
  \end{verbatim}
    ger t ex 
    \begin{equation}
      \label{eq:first}
      f(x) = x^2 + 2x + 9
    \end{equation}
    och refererar man sen till etiketten \texttt{eq:first} med 
    \verb!Eq.~\ref{eq:first}! gör \LaTeX\ om detta till 
    \begin{quote}
      Eq.~\ref{eq:first}
    \end{quote}
    Den ''våg'' (\verb|~|) som syns i koden kallas \emph{tilde} och är
    ett fast mellanslag, ett slags klister som beordrar \LaTeX\ att
    behandla det ihopklistrade som ett ord. Detta används för att
    hindra radbrytning mellan Eq. och siffran \ref{eq:first}. I just
    detta sammanhang får radbrytning inte ske, och \LaTeX\  måste då
    upplysas om det. 

  \item \verb|\cdot| ger en centrerad punkt i matematik. En sådan
    brukar användas för att visa multiplikation mellan två tal eller
    där man annars vill trycka på att det verkligen är
    multiplikation. Annars ska man aldrig använda en punkt för
    multiplikation. Alltså, man skriver $f(x)g(x)$ och inte $f(x)\cdot
    g(x)$. Däremot måste man skriva $3\cdot 4$ för annars blir det
    $34$, dvs ett tal och inte en multiplikation. 

  \item Tecknet \verb|^| ger ''upphöjt i'', dvs en exponent i
    matematisk text. Exempel: Att skriva \verb|$x^{f(x)}$| ger
    $x^{f(x)}$.  Notera användningen av klammerparenteser för att
    grupera ihop hela exponenten på samma sätt som de tidigare höll
    samman större index (se sidan \ref{sec:maas}). Har man en enkel
    exponent behövs inga klammerparenteser. För att få $x^2$ räcker
    det att skriva \verb|$x^2$|. 

  \item \verb|\frac| används för bråk. Exempel:
    \verb|\frac{L_{n-1}}{3}| ger, om det placeras i en
    \texttt{displaymath}, 
    \begin{displaymath}
      \frac{L_{n-1}}{3}
    \end{displaymath}
    \verb|\frac| passar ofta mindre bra i löptext, då allt kan se
    högst ihoptryckt ut: $\frac{L_{n-1}}{3}$. Då kan man ofta få
    finare text genom att använda ett helt vanligt snedstreck: 
    $L_{n-1}/3$

  \item \verb|\left| och \verb|\right| ger storleksanpassade
    parenteser. Skriver man \verb|( \frac{4}{3} )| får man 
    \begin{displaymath}
      ( \frac{4}{3} )
    \end{displaymath}
    medan \verb|\left( \frac{4}{3} \right)| ger det betydligt snyggare 
    \begin{displaymath}
      \left( \frac{4}{3} \right)
    \end{displaymath}
    Istället för parenteser kan man använda annat t ex
    klammerparenteser eller hakparenteser. Observera att \verb|\left|
    och \verb|\right| alltid förekommer i par. 

  \item \verb|\infty| ger tecknet $\infty$, oändligheten. 

  \item Förkortningen i.e., dvs ''that is'', skrivs \verb| i.e.\ | med
    ett mellanslag direkt efter bakåtsnedstrecket. 

  \item \verb|\sqrt| ger kvadratroten ur något. \verb|$\sqrt{x}$| blir
    $\sqrt{x}$.  

  \item \verb|\ldots| ger tre punkter på textens baslinje och används
    för att markera att något utelämnats. 

  \item Omgivningen \verb|equation*| fungerar som
    \texttt{displaymath}. 

  \item \verb|align*| är en omgivning för ekvationer som går över
    många led och rader. 

  \item \verb|\sum| ger ett summatecken i matematisk text. Skriver man 
    \verb| \sum_{k=1}^n b_k| får man t ex $ \sum_{k=1}^n b_k$ i
    löptext och  
    \begin{displaymath}
       \sum_{k=1}^n b_k
    \end{displaymath}
    i fri text. 

  \item \verb|\lim| används för gränsvärden. Skriver man t ex
    \begin{center}
      \verb!\lim_{x \to \infty} \frac{1}{x} = 0!
    \end{center}
    blir det $\lim_{x \to \infty} \frac{1}{x} = 0$ i löptext och 
    \begin{displaymath}
      \lim_{x \to \infty} \frac{1}{x} = 0
    \end{displaymath}
    i en \texttt{displaymath}. 

  \item \verb|\to| ger en högerpil.

  \item \verb|thebibliography| är en omgivning som bygger upp en
    referenslista där varje referens anges med hjälp av kommandot
    \texttt{bibitem}.  

  \item \verb|\textbf| gick vi slutligen igenom på sidan
    \pageref{sec:textbf} i början av detta dokument.   
\end{itemize}


\end{document}
