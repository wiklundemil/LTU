\documentclass[a4paper,12pt]{article}
\usepackage[swedish]{babel}
\usepackage[utf8]{inputenc}
\usepackage{amsmath, amsthm, amssymb}
\usepackage[a4paper,includeheadfoot,margin=2.54cm]{geometry}
\usepackage{enumerate}
\renewcommand{\familydefault}{\sfdefault}


\title{D0015E Datateknik och ingenjörsvetenskap \\
       Uppgift i moment L om yrkesrollen}
%
\author{Emil Wiklund}
%
\date{\today}



\begin{document}


\maketitle


% Kom ihåg att ändra författarnamn ovan till ditt eget namn
% Ändra inte någon avsnittsrubrik


\section*{Svar på uppgift 1}


% Skriv svaren ett i taget nedan. 

\begin{enumerate}
  \item En aledning till varför företag vinner på att jobba med 
  jämstdäldhetsfrågor är pågrund utav det lockar fler kvinnor 
  till mansdominerade arbetdsplatser och fler män till arbetsplatser
  som är kvinno dominanta. Utifrån detta kan företagen locka fler
  människor med olika bakgrunder och utifrån detta skapa en bättre
  och effektivare arbetsplats. 
 %
 \item En annan anledning till varför förtagen tjänar på att 
  arbeta med jämstäldhetsfrågor är pågrund utav att det ökar
  kompetensen på företaget. Det som menas med detta är att en mer 
  blandad bas av kvinnor och män ger upphov till olika erfarenheter, 
  olika tänk som är nyttigt för företagen. Med detta kommer företaget
  att kunna växa och utvecklas. 
  %
  \item Ytterligare en anledning till varför företagen vinner på att arbeta med
  jämstäldhetsfrågor är pågrund utav att dom kommer att skapa en trivsel 
  för inte bara män utan även för kvinnor. Detta pågrund utav att det inte 
  ska bli en macho eller kvinno kultur som är skadosam på dess anstälda. 
\end{enumerate}


\section*{Svar på uppgift 2}


% Skriv ditt svar på uppgiften här.

\begin{enumerate}
  \item Ett förslag skulle vara att kunna låta samtliga killar på arbetsplatsen 
  komma på $x$ antal frågor och sedan låta samtliga tjejer på arbetsplatsen ge
  förslag på lika många frågor som killarna. Om man låter de två könen ge lika
  många förslag så kommer det alltså att bli rättvist för båda parter. Lika många
  förslag vare sig det är fler killar eller fler tjejer på arbetsplatsen kommer 
  att tas upp.
  %
  \item Ett annat förslag skulle vara att kunna ha ett möte någon gång i 
  månaden där man får ge förslag mot förbättringar. Utifrån detta skulle man 
  kunna få fram idér från båda parter. Sedan kan dessa tas in i praktiken och 
  se hur dessa påverkar arbetsplatsen. Om det skulle påverka arbetsplatsen 
  negativt så kan man gå tillbaka och försöka med ett annat förslag.
 %
 \item Ännu ett förslag som skulle kunna hjälpa arbetsplatser att få det mera
  likvärdigt är att man skulle ha ett slags samtal med lika många kvinnor som män.
  Utifrån detta skulle man kunna ändra på hur arbetsplatsen, detta skulle man 
  okcså kunna göra på ett anonymt sätt. Detta är kanske till och med bättre än 
  att prata öppet inför grupp. Detta då man kanske inte alltid vågar att säga 
  det man vill eftersom man har en slags press på sig då. 
\end{enumerate}
  
\section*{Svar på uppgift 3}



Om vi exemeplvis tar en sjuksköterska, inom denna yreksroll skulle
vi säga att denna som person skulle vara kunnig med personer och vara kunnig 
inom medecin och även kunna hantera stress på ett positivt sätt. Om vi tänker 
oss ingejörer skulle man tänka att en ingenjör ska kunna lösa problem och att de
även ska vara innovativa och kunna tänka ut nya lösningar till problem. Sedan
så skulle man även kunna tänka sig att en ingejör är mycket noga när denna arbetar.
En ingenjör skulle tänka sig alla möjliga situationer där lösningen skulle kunna verka, 
och även tänka till på hur effektiv lösningen är, hur mycket den kostar samt hur 
säker lösningen skulle vara. Sammanfattar man detta skulle en ingejörs önskevärda 
karaktärsegenskaper vara att se problem och lösningar till problemen samt vara 
innovativa och även mycket nogranna.



% Skriv ditt svar på uppgiften här.

% Om referens till kompendiet behövs så använd denna: \cite{komp}


\section*{Svar på uppgift 4}


\begin{enumerate}
  \item Ett skäl för varför en ingejör skall kunna ha en legitimation 
  skulle kunna vara för att om en ingejör gör ett fel medvetet i exempelvis en bro 
  konstruktion skulle detta kunna resultera i att många personer utsätts för fara. Att en 
  ingejör gör ett medvetet fel som kan utsätta personer för fara skulle kunna vara
  en anledning till varför denne i detta fall skulle få sin legitimation återkallad. 
  %
  \item Skäl emot att en ingenjör skulle kunna ha en legitimation
  skulle kunna vara pågrund utav att det finns så pass många ingenjörsroller skulle det
  vara svårt att kunna dra gränsen mellan det arbete som en ingejör skall göra och
  inte göra. Eftersom en ingejör gör så pass många olika saker skulle det vara svårt att bedömma
  om denne gör något som den får eller inte får göra. Därav att det skulle vara svårt att
  återkalla dennes legitimation. 
  %
  \item Min egna ståndpunkt inom detta skulle vara att det skulle vara bra med
  en sorts legitimation just pågrund utav att denne inte gör något som är utsätter folk
  för fara. Lösningen skulle kunna vara att skapa en legitimation där det finns extrema 
  fall som om individen genomför ett sorts arbete där man som ingejör beter sig, vårdslöst
  eller medvetet gör fel för att skada andra kunna få sin legitimation indraget.
\end{enumerate}
% Skriv ditt svar på uppgiften här.

\newpage
\section*{Svar på uppgift 5}


  Uppgiften som valdes att svara på är uppgift 6. Det som sägs är att du är anställd 
  och på din arbetsplats har du fått uppgiften att utveckla en radardetektor som kan 
  upptäcka polisens radarövervakning. Frågan är vad du som anställd skulle göra då.
  Den anställda ska först och främst kolla upp om detta är lagligt att göra i det landet
  som denne beffiner sig i. Därefter så bör den som fått uppgiften kontrollera om det
  det verkligen var en radardetektor som var uppgiften. Om det var så att det var 
  just det som ingjören fick i uppgift att utveckla så borde denna uppmana chefen att
  det som ska utvecklas anses som olagligt. Däremot skulle det vara lagligt att tillverka 
  detta skulle denne påbörja utvecklingen. Om man skulle utgå från Civilingejörsförbundets
  hederkodex kan vi se att Enligt den första punkten i kodexen säger att en ingejör ska
  känna ett personligt ansvar för tekniken, människa miljö och sammhälle. Utifrån detta
  så bör denne som utvecklar radardetektorn kontrollera om det faktist är lagligt med 
  en sådan. Om det inte är så ska denne alltså som tidigare nämnst anmäla till sin chef
  och förklara att denna är olaglig. Detta kan även kopplas till punkt 3 från
  kodexen där man ska ingengör ska teknikens möjligheter samt risker. 
  
% Skriv ditt svar på uppgiften här.


\begin{thebibliography}{99}
  \bibitem{komp} Sven Ove Hansson. \emph{Teknik och etik}.
    Kompendium, 2009. \\
    URL: \verb|http://www.infra.kth.se/~soh/tekniketik.pdf| \\
    Läst 2020-10-12. 
  \end{thebibliography}
  %
\end{document}
