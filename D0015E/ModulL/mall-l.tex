\documentclass[a4paper,12pt]{article}
\usepackage[swedish]{babel}
\usepackage[utf8]{inputenc}
\usepackage{amsmath, amsthm, amssymb}
\usepackage[a4paper,includeheadfoot,margin=2.54cm]{geometry}
\usepackage{enumerate}
\renewcommand{\familydefault}{\sfdefault}


\title{D0015E Datateknik och ingenjörsvetenskap \\
       Uppgift i moment L om yrkesrollen}
%
\author{Emil Wiklund}
%
\date{\today}



\begin{document}


\maketitle


% Kom ihåg att ändra författarnamn ovan till ditt eget namn
% Ändra inte någon avsnittsrubrik


\section*{Svar på uppgift 1}


% Skriv svaren ett i taget nedan. 

\begin{enumerate}
  \item Nytta/fördel 1 + hur den kan uppnås
  \item Nytta/fördel 2 + hur den kan uppnås
  \item Nytta/fördel 3 + hur den kan uppnås
\end{enumerate}


\section*{Svar på uppgift 2}


% Skriv ditt svar på uppgiften här.

\begin{enumerate}
  \item Förslag 1
  \item Förslag 2
  \item Förslag 3
\end{enumerate}
  
\section*{Svar på uppgift 3}


% Skriv ditt svar på uppgiften här.

% Om referens till kompendiet behövs så använd denna: \cite{komp}


\section*{Svar på uppgift 4}


% Skriv ditt svar på uppgiften här.


\section*{Svar på uppgift 5}


% Skriv ditt svar på uppgiften här.


\begin{thebibliography}{99}
  \bibitem{komp} Sven Ove Hansson. \emph{Teknik och etik}.
    Kompendium, 2009. \\
    URL: \verb|http://www.infra.kth.se/~soh/tekniketik.pdf| \\
    Läst 2020-10-12. 
  \end{thebibliography}
  %
\end{document}
