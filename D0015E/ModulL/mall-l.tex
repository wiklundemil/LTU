\documentclass[a4paper,12pt]{article}
\usepackage[swedish]{babel}
\usepackage[utf8]{inputenc}
\usepackage{amsmath, amsthm, amssymb}
\usepackage[a4paper,includeheadfoot,margin=2.54cm]{geometry}
\usepackage{enumerate}
\renewcommand{\familydefault}{\sfdefault}


\title{D0015E Datateknik och ingenjörsvetenskap \\
       Uppgift i moment L om yrkesrollen}
%
\author{Emil Wiklund}
%
\date{\today}



\begin{document}


\maketitle


% Kom ihåg att ändra författarnamn ovan till ditt eget namn
% Ändra inte någon avsnittsrubrik


\section*{Svar på uppgift 1}


% Skriv svaren ett i taget nedan. 

\begin{enumerate}
  \item En aledning till varför företag vinner på att jobba med 
  jämstdäldhetsfrågor är pågrund utav det lockar fler kvinnor 
  till mansdominerade arbetdsplatser och fler män till arbetsplatser
  som är kvinno dominanta. 
  \item En annan anledning till varför förtagen tjänar på att 
  arbeta med jämstäldhetsfrågor är pågrund utav att det ökar
  kompetensen på företaget. Det som menas på är att en mer blandad
  bas av kvinnor och män ger upphov till olika erfarenheter, olika tänk
  som är nyttigt för företagen. 
  \item Ännu en anledning som bygger på den föregående nyttan är
  att företgen kommer att utvecklas för en social hållbarhet.
\end{enumerate}

5:00 i videon
1.Attraktiva arbetsplatser
2.Kompetens
3.Utveckling och förutsättning för social hållbarhet

upg2

1.
2.
3.

\section*{Svar på uppgift 2}


% Skriv ditt svar på uppgiften här.

\begin{enumerate}
  \item Förslag 1
  \item Förslag 2
  \item Förslag 3
\end{enumerate}
  
\section*{Svar på uppgift 3}


% Skriv ditt svar på uppgiften här.

% Om referens till kompendiet behövs så använd denna: \cite{komp}


\section*{Svar på uppgift 4}


% Skriv ditt svar på uppgiften här.


\section*{Svar på uppgift 5}


% Skriv ditt svar på uppgiften här.


\begin{thebibliography}{99}
  \bibitem{komp} Sven Ove Hansson. \emph{Teknik och etik}.
    Kompendium, 2009. \\
    URL: \verb|http://www.infra.kth.se/~soh/tekniketik.pdf| \\
    Läst 2020-10-12. 
  \end{thebibliography}
  %
\end{document}
