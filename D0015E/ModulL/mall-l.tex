\documentclass[a4paper,12pt]{article}
\usepackage[swedish]{babel}
\usepackage[utf8]{inputenc}
\usepackage{amsmath, amsthm, amssymb}
\usepackage[a4paper,includeheadfoot,margin=2.54cm]{geometry}
\usepackage{enumerate}
\renewcommand{\familydefault}{\sfdefault}


\title{D0015E Datateknik och ingenjörsvetenskap \\
       Uppgift i moment L om yrkesrollen}
%
\author{Emil Wiklund}
%
\date{\today}



\begin{document}


\maketitle


% Kom ihåg att ändra författarnamn ovan till ditt eget namn
% Ändra inte någon avsnittsrubrik


\section*{Svar på uppgift 1}


% Skriv svaren ett i taget nedan. 

\begin{enumerate}
  \item En aledning till varför företag vinner på att jobba med 
  jämstdäldhetsfrågor är pågrund utav det lockar fler kvinnor 
  till mansdominerade arbetdsplatser och fler män till arbetsplatser
  som är kvinno dominanta. 
  \item En annan anledning till varför förtagen tjänar på att 
  arbeta med jämstäldhetsfrågor är pågrund utav att det ökar
  kompetensen på företaget. Det som menas på är att en mer blandad
  bas av kvinnor och män ger upphov till olika erfarenheter, olika tänk
  som är nyttigt för företagen. 
  \item Ännu en anledning till varför företagen vinner på att arbeta med
  jämstäldhetsfrågor är pågrund utav att dom kommer att skapa en trivsel 
  för inte bara män utan även för kvinnor. Detta pågrund utav att det inte 
  ska bli en machokultor som är skadosam på dess anstälda. 
\end{enumerate}


\section*{Svar på uppgift 2}


% Skriv ditt svar på uppgiften här.

\begin{enumerate}
  \item Ett förslag skulle vara att kunna låta samtliga killar på arbetsplatsen 
  komma på $x$ antal frågor och sedan låta samtliga tjejer på arbetsplatsen ge
  förslag på lika många frågor som killarna. Om man låter de två könen ge lika
  många förslag så kommer det alltså att bli rättvist för båda parter. Lika många
  förslag vare sig det är fler killar eller fler tjejer på arbetsplatsen kommer 
  att tas upp.
  \item Ett annat förslag skulle vara att kunna ha ett möte någon gång i 
  månaden där man får ge förslag mot förbättringar. Utifrån detta skulle man 
  kunna få fram idér från båda parter. Sedan kan dessa tas in i praktiken och 
  se hur dessa påverkar arbetsplatsen. Om det skulle påverka arbetsplatsen 
  negativt så kan man gå tillbaka och försöka med ett annat förslag.
  \item Ännu ett förslag som skulle kunna hjälpa arbetsplatser att få det mera
  likvärdigt är att man skulle ha ett slags samtal med lika många kvinnor som män.
  Utifrån detta skulle man kunna ändra på hur arbetsplatsen fungerar även på ett
  annonymt. Detta är kanske till och med bättre än att prata öppet inför grupp.
  Detta då man kanske inte alltid vågar att säga det man vill eftersom man har 
  en slags press. 
\end{enumerate}
  
\section*{Svar på uppgift 3}



Om vi exemeplvis tar en sjuksköterska skulle vi då säga att denna yreksroll skulle
vi säga att denna som person skulle vara kunnig med personer och vara 
problemlösare
inovativ
nytänkande
sammarbets
perfektionist (Det ska vara effektvt och bra)

FÖrklara hur du vill att en ingejör ska vara, hur ska en kapten vara och hur ska en sjukskötersska vara exempelvis?
% Skriv ditt svar på uppgiften här.

% Om referens till kompendiet behövs så använd denna: \cite{komp}


\section*{Svar på uppgift 4}


% Skriv ditt svar på uppgiften här.


\section*{Svar på uppgift 5}


% Skriv ditt svar på uppgiften här.


\begin{thebibliography}{99}
  \bibitem{komp} Sven Ove Hansson. \emph{Teknik och etik}.
    Kompendium, 2009. \\
    URL: \verb|http://www.infra.kth.se/~soh/tekniketik.pdf| \\
    Läst 2020-10-12. 
  \end{thebibliography}
  %
\end{document}
