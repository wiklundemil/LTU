\documentclass[12pt,a4paper]{article}
%\usepackage{showframe}
\usepackage[utf8]{inputenc}
\usepackage{graphicx}
\usepackage{amsmath, amsthm, amssymb}
\usepackage[a4paper,includeheadfoot,margin=2.54cm]{geometry}
\newtheorem{theorem}{Theorem}

\begin{document}	
	\section{The Koch Snowflake}
		The \emph{Koch snowflake}, one of the first fractals, is based on work by the Swedish mathematician Helge von Koch~\cite{koch}.
		It is what we get if we start with an equilateral triangle\begin{figure}[h] \label{koch}
			  \centering
			  \includegraphics[width=10cm]{snowflake.jpg}
			  \caption{The initial equilateral triangle and the refinement of the Koch snowflake after
				     one, two, and three iterations.}
			\end{figure}		
			and repeat the following an infinite number of times:
			\begin{quote}
				 \textit{Divide all line segments into three segments of equal length. Then draw, for
				          	   each middle line segment, an equilateral triangle that has the middle segment
					   as its base and points outward. Finally, remove all middle segments.}
			\end{quote}
		Figure \ref{koch} shows the first iterations in the construction.\marginpar{(Original)}
	\subsection{Two porperties}
	\begin{theorem}
	 	\textit{The Koch snowflake has infinite length.}
	\end{theorem}
	\begin{proof}
		Let $\Delta$  denote a triangle, with side length \textit{s}, on which we base the construction 
		of~a~snowflake. Let $N_i$ denote the number of line segments, and  $L_i$ the length of the
		segments, in iteration $i$ of the construction. Then    	
		\begin{displaymath}
			  N_n =
			    \begin{cases}
			      3             & \text{if $n=0$ (i.e.\ before any iterations), and} \\
			     4 N_{n-1}		& \text{otherwise,}
			    \end{cases}
		\end{displaymath}
		which solves to 
		\begin{equation}
		 	\label{eq:1}
		  	 N_n=3\cdot 4^n,
		\end{equation}
		while  
		\begin{equation}
		  \label{eq:5}
		   L_n=\frac{L_{n-1}}{3}=\frac{L_{n-2}}{3^2}=\frac{L_{n-3}}{3^3}=\ldots=\frac{L_0}{3^n}=\frac{s}{3^n}.
		\end{equation}
		From Eqs.~\ref{eq:1} and \ref{eq:5}, the total length 
		\begin{displaymath} 
			  N_nL_n=3\cdot4^n\frac{s}{3^n}=3s\frac{4^n}{3^n}=3s\left(\frac{4}{3}\right)^n.
		\end{displaymath}	
		Since 4/3 $>$ 1, it follows that $ N_{n}L_{n}$ tends to infinity as n $\to \infty$, i.e. the Koch snowflake has infinite length.
	\end{proof}
	\begin{theorem}
	The Koch snowflake has finite area.
		\begin{proof}
			In an iteration, a triangle is added on each line segment of the previous iteration.
			So, in iteration \textit{n}, the number of new triangles $T_n=N_{n-1}$, which, by
			Eq.~\ref{eq:1}, can be simplified to 
			  \begin{equation}
			   	\label{eq:2}
			   	T_n =\frac{3}{4} \cdot 4^n .
			  \end{equation}
			   The area $a_n$ of each such triangle, with the exception of the area
			 \begin{displaymath}
				a_0=\frac{\sqrt{3}}{4}s^2
			 \end{displaymath}
			 of $\Delta$, the initial equilateral triangle, is one ninth of the area of a triangle added in iteration
			 \textit{n}$- 1$, or				
			\begin{equation}
				\label{eq:3}
				a_n=\frac{a_{n-1}}{9}=\frac{a_{n-2}}{9^2}=\ldots=\frac{a_{1}}{9^{n-1}}=\frac{a_{0}}{9^n}
			 \end{equation}
			 This means that in iteration \textit{n}, by Eqs.~\ref{eq:2} and \ref{eq:3}, the area of all added triangles 
			 \begin{displaymath}
				b_n=T_ {n}a_{n}=\left(\frac{3}{4}\cdot4^n\right)\left(\frac{a_0}{9^n}\right)=\frac{3a_0}{4}\left(\frac{4}{9}\right)^n.
			 \end{displaymath}\marginpar{(Original)}
			   All in all, after iteration \textit{n}, the total area
			  \begin{align*}
			           A_n &= a_0 + \sum_{k=1}^n b_k \\
			           &=a_0\left(1+\frac{3}{4}\sum_{k=1}^n\left(\frac{4}{9}\right)^k \right) \\
				&=a_0\left(1+\frac{1}{3}\sum_{k=0}^{n-1}\left(\frac{4}{9}\right)^k \right) \\
				&=a_0\left(1+\frac{3}{5}\left(1-\left(\frac{4}{9}\right)^n\right)\right)\\
				&=\frac{a_0}{5}\left(8-3\left(\frac{4}{9}\right)^n \right).
			   \end{align*}
			  Now, since
			\begin{displaymath}
			   \lim_{n \to \infty}3\left(\frac{4}{9}\right)^n=0,
			\end{displaymath}
			 it follows that $\lim_{n \to \infty}A_n=8a_0/5$, i.e.\ the Koch snowflake has finite area.  
		\end{proof}
	\end{theorem}
	
	
	\begin{thebibliography}{99}
	  \bibitem{koch} Helge von Koch. 
	    \emph{Sur une courbe continue sans tangente, obtenue par une
	    construction géométrique élémentaire.}, 
	    Arkiv för matematik, astronomi och fysik, 
	    Kungliga Vetenskapsakademien. 
	    \textbf{1}, 681-702, 1904.
	\end{thebibliography}
\end{document}