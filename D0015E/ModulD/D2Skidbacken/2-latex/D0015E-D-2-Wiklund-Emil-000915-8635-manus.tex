\documentclass[a4paper,12pt]{article}
\usepackage[swedish]{babel}
\usepackage[utf8]{inputenc}
\usepackage{amsmath, amsthm, amssymb}
% Ändra INTE nästa rad (säger var texten ska typsättas)
\usepackage[a4paper,includeheadfoot,margin=2.54cm]{geometry}
% Ändra INTE nästa rad (som lägger till radnummer till vänster)
\usepackage[left]{lineno}


% Ändra INTE raderna nedan
% Koden är från https://tex.stackexchange.com/questions/43648/
% Den fixar radnumrering av text i närvaro av matematikomgivningar
\newcommand*\patchAmsMathEnvironmentForLineno[1]{%
  \expandafter\let\csname old#1\expandafter\endcsname\csname #1\endcsname
  \expandafter\let\csname oldend#1\expandafter\endcsname\csname end#1\endcsname
  \renewenvironment{#1}%
     {\linenomath\csname old#1\endcsname}%
     {\csname oldend#1\endcsname\endlinenomath}}% 
\newcommand*\patchBothAmsMathEnvironmentsForLineno[1]{%
  \patchAmsMathEnvironmentForLineno{#1}%
  \patchAmsMathEnvironmentForLineno{#1*}}%
\AtBeginDocument{%
\patchBothAmsMathEnvironmentsForLineno{equation}%
\patchBothAmsMathEnvironmentsForLineno{align}%
\patchBothAmsMathEnvironmentsForLineno{flalign}%
\patchBothAmsMathEnvironmentsForLineno{alignat}%
\patchBothAmsMathEnvironmentsForLineno{gather}%
\patchBothAmsMathEnvironmentsForLineno{multline}%
}

% Ändra INTE nästa rad (gör så radnummer skrivs med fet stil)


\renewcommand\linenumberfont{\normalfont\bfseries\small}


\title{Skidbacken}
%
\author{Emil Wiklund\thanks{email: \texttt{emiwik-9@student.ltu.se}}\\  
        ~ \\
        Luleå tekniska universitet \\ 
        971 87 Luleå, Sverige}
%          
\date{\today}


\begin{document}


% Ändra INTE nästa rad (aktiverar utskrift av radnummer)
\linenumbers
\maketitle


\begin{abstract}
Sammanfattningsvis handlar denna raport om vilken derivatra olika
punkter på en graf erhåller samt lösandet av punkter. Rapporten ger
lösningar på tre problem. 

Det första problemen handlar om att beräkna derivatan i en punkt 
med ett givet $x$-värde. Lösningen visar på ett korrekt sätt att lösa
denna typ av uppgift som också ger oss ett exakt svar på vad för
derivata den specifika punkten erhåller.

Det andra problemet handlar om att beräkna den punkt där derivatan
är som störst. Detta problem löser vi med hjälp av andraderivata 
tillsammans med användning av kedjeregeln och kunskap angående 
derivata samt andraderivata.

Det tredje problemet handlar om beräkning av en okänd variabel.
Den okända variabeln blir känd genom användning av ett givet värde 
ur en punkt  samt användningen av ett utryck från en föregående 
deluppgift.
\end{abstract}


\section{Introduktion}
\label{sec:introduktion}


Denna rapport går ut på att visa lösningar ur uppgiften \emph{skidbacken}.
I uppgiften finns det tre stycken deluppgifter vilket rapporten går igenom.
Dessa handlar om begreppen \emph{derivata} och \emph{lutning}.
Det uppgiften går ut på är hur man räknar ut \emph{derivatan} vid ett vist 
tillfälle. Deluppgift ett går igenom backens \emph{lutning} i en viss punkt.
Deluppgift två går igenom lösandet av backens brantaste punkt den tredje 
och sista deluppgiften går igenom lösandet av variabeln $a$.
\newpage


\subsection{Derivata och lutning}


Dokumentet nämner begreppet \emph{derivata} och \emph{lutning} och då är
det viktigt att den som läser detta kan förstå innebörden av dessa. Dessa är 
kopplade till varandra och när \emph{derivata} nämns menar man på 
\emph{lutningen}. \emph{Derivatan} eller \emph{lutningen} tyder på 
förändringshastighet i en viss tidpunkt.


Om vi exempelvis har två olika punkter. Punkten $(x, f(x))$ och punkten 
$(x+h, f(x+~h))$ kan man dra en linje mellan dessa, en \emph{sekant}
mellan punkterna. Med \emph{sekant} menar man på medellutningen till 
två punkter. I det här fallet punkterna $(x, f(x))$ och $(x+h, f(x+h))$. 
Skulle man låta $h$  gå mot $0$ så kommer sekantens \emph{lutning} att
tillslut övergå till en tangent. Då kan vi teckna ett utryck för sekantens 
\emph{lutning}:
%
\begin{align*}
  k=\frac{\Delta y}{\Delta x}=\frac{f(x+h)-f(x)}{h}
\end{align*}
%
Om vi låter $h$ gå mot $0$
%
\begin{align*}
  \lim_{h\to 0}\frac{f(x+h)-f(x)}{h}
\end{align*}
%
får vi tangentens \emph{lutning} genom att 
räkna ut gränsvärdet i punkten $(x, f(x))$.


\section{Backens lutning}
\label{sec:uppg1}


Uppgiftens lösande sker på följande sätt:
%
\begin{quote} 
  Bestäm backens lutning för $x = 0.8$.
\end{quote}
%
\emph{Lösning:} Derivera $y=0.5e^{-x^2}$ för att få ut funktionen för 
lutningen, funktionen för derivatan ser ut: $y'=xe^{-x^2}$
Därefter sätter vi in värdet $x=0.8$ i funktionen för lutningen
%
\begin{align*}
  y'(0.8)=0.8e^{-0.8^2}
\end{align*}
%
Insättningen av $0.8$ ger lutningen värdet $\approx(-0.42)$.Vilket är svaret
på deluppgiften. 
%


\section{Backens brantaste punkt}
\label{sec:uppg2}


Uppgiftens lösande sker på följande sätt:
%
\begin{quote}
  Ställ upp en ekvation för bestämning av $x$-värdet i den punkt där backar 
  med en sådan banprofil är brantast.
\end{quote}
%
\emph{Lösning:} Det uppgiften frågar efter är $x$-värdet i den punkten där 
backen är som brantast. Vi vet att backens \emph{lutning} är som störst 
där $y'$ är störst. Det vill säga där $y''$ är lika med~0.\newpage
%
Vi beräknar och skriver om $y'$:
%
\begin{align*}
  &y=0.5\cdot e^{-ax^2}\\
  &y'=-ax\cdot e^{-ax^2}\\
  &y'=-\frac{ax}{e^{ax^2}}
\end{align*}
%
Här väjer vi att använda \emph{kedjeregeln} då vi utifrån $y'$ kan få ut 
fyra stycken funktioner. Dessa kallar vi för$f(x)=-ax$, $f'(x)=-a$, $g(x)=e^{-x^2}$ och $g'(x)=-2ax\cdot e^{-x^2}$. 
Genom \emph{kedjeregeln} kan vi få ut \emph{andraderivatan}. Vi fortsätter
genom att sätta in de kända värdena och ställer 
upp en ekvation enligt \emph{kedjeregeln}:
%
\begin{align*}
  \label{eq:1}
  y''=f'(x)g(x)-f(x)g'(x)
  %Enklare att se med en sepparation
  =-a\cdot e^{ax^2}+ax\cdot 2ax\cdot e^{ax^2}
\end{align*}
%
Därefter eftersom \emph{lutningen} var är störst när \emph{andraderivatan}
är lika med 0 ger vi $y''$ värdet 0, efter detta förenklar vi ekvationen:
%
\begin{align*}
  &-a\cdot e^{ax^2}+ax\cdot 2ax\cdot e^{ax^2}=0\\
  &-a\cdot e^{-ax^2}+2a^2x^2\cdot e^{-ax^2}=0\\
  &a\cdot e^{-ax^2}(2ax^2-1)=0
\end{align*}
%
Utifrån detta använder vi oss utav nollproduktsmetoden som ger följande
$ae^{ax^2}>0$. Om vi utgår ifrån detta vilkor kan vi få ut ett värde för 
$x$.
%
\begin{align*}
  &2ax^2-1=0\\
  &2ax^2=1\\
  &x^2=\frac{1}{2a}\\
  &x=\sqrt{\frac{1}{2a}}
\end{align*}
%
Detta ger oss svaret till uppgiften. $x=\sqrt{\frac{1}{2a}}$ är det $x$-värde
i den punkten där \emph{derivatan} är som störst. 
Det vill säga där backen är som brantast.


\section{Lösning av variabel}
\label{sec:uppgN}


Uppgiftens lösande sker på följande sätt:
%
\begin{quote}
  Bestäm $a$ så att backen är brantast för $x=1.0$.
\end{quote}
%
\emph{Lösning:} Uppgiften går att lösa genom  användandandet av $x$-värdet
från föregående deluppgift.
%
\begin{displaymath} 
  x=\sqrt{\frac{1}{2a}}
\end{displaymath}
%
Det uppgiften frågar efter är det värde som $a$ kommer att ha då $x=1.0$
Genom att sätta in $x$-värdet i utrycket ovan
ger det oss möjligheten att ställa upp en ekvation som går att lösa på 
följande sätt:
%
\begin{align*} 
  &1=\sqrt{\frac{1}{2a}}\\
  &1^2=\frac{1}{2a}\\
  &2a=1\\
  &a = \frac{1}{2}=0.5.\label{eq:2}
\end{align*}
%
Genomförd ekvation ger oss $a$-värdet 0.5 vilket är svaret på den sista deluppgiften. 


\section{Diskussion}
\label{sec:disk}


I denna rapport har vi löst tre stycken deluppgifter relaterade till derivata genom
användning av \emph{kedjeregeln} samt andraderivata.

Att lösa denna typen av problem är enkelt i denna uppgift. Däremot hade uppgiften
varit större, exempelvis bestått av svårare utryck eller ett värde i en punkt med 
fler decimaler kunde uppgiftens svårighetsgrad ökat betydligt. Men utifrån denna
uppgift kan man skapa sig en förstelse kring hur man löser dessa typer av problem,
vilket gör det enklare genom en stärkt försteålse lösa svårare problem.
%
\end{document}