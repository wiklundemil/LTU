\documentclass[a4paper,12pt]{article}
\usepackage[swedish]{babel}
\usepackage[utf8]{inputenc}
\usepackage{amsmath, amsthm, amssymb}
% Ändra INTE nästa rad (säger var texten ska typsättas)
\usepackage[a4paper,includeheadfoot,margin=2.54cm]{geometry}
% Ändra INTE nästa rad (som lägger till radnummer till vänster)
\usepackage[left]{lineno}


% Ändra INTE raderna nedan
% Koden är från https://tex.stackexchange.com/questions/43648/
% Den fixar radnumrering av text i närvaro av matematikomgivningar
\newcommand*\patchAmsMathEnvironmentForLineno[1]{%
  \expandafter\let\csname old#1\expandafter\endcsname\csname #1\endcsname
  \expandafter\let\csname oldend#1\expandafter\endcsname\csname end#1\endcsname
  \renewenvironment{#1}%
     {\linenomath\csname old#1\endcsname}%
     {\csname oldend#1\endcsname\endlinenomath}}% 
\newcommand*\patchBothAmsMathEnvironmentsForLineno[1]{%
  \patchAmsMathEnvironmentForLineno{#1}%
  \patchAmsMathEnvironmentForLineno{#1*}}%
\AtBeginDocument{%
\patchBothAmsMathEnvironmentsForLineno{equation}%
\patchBothAmsMathEnvironmentsForLineno{align}%
\patchBothAmsMathEnvironmentsForLineno{flalign}%
\patchBothAmsMathEnvironmentsForLineno{alignat}%
\patchBothAmsMathEnvironmentsForLineno{gather}%
\patchBothAmsMathEnvironmentsForLineno{multline}%
}


% Ändra INTE nästa rad (gör så radnummer skrivs med fet stil)
\renewcommand\linenumberfont{\normalfont\bfseries\small}


\title{Skidbacken}
%
\author{Emil Wiklund\thanks{email: \texttt{emiwik-9@student.ltu.se}}\\  
        ~ \\
        Luleå tekniska universitet \\ 
        971 87 Luleå, Sverige}
%          
\date{\today}


\begin{document}


% Ändra INTE nästa rad (aktiverar utskrift av radnummer)
\linenumbers


\maketitle


\begin{abstract}
%  I denna sammanfattning skriver jag helt kort ner vad resten av
% dokumentet handlar om. Här kan det passa att beskriva vilken
%  uppgiften/uppgifterna är och i stora drag hur den/de lösts.
%  Man ska efter att ha läst sammanfattning förstås vad dokumentet
%  handlar om, även om man inte vet i detalj. 

   
\end{abstract}


\section{Introduktion}
\label{sec:introduktion}


%Här kan man skriva massor om uppgiften, dess bakgrund, varför det är
%viktiga att lösa det osv men för oss räcker det med att introducera
%den. Ett bra sätt att skriva på är att sen behandla varje problem i
%egna avsnitt, där man inleder med att beskriva problemet noga innan
%man visar hur man kan lösa det.


%Faktum är att det blir lite krystat att skriva en liten rapport om hur
%man löser en uppgift som denna men vi gör det som en övning i \LaTeX. 

\subsection{Derivata och lutning}


Dokumentet nämner termen \emph{derivata} och ordet \emph{lutning} och då är det viktigt att den som läser detta kan förstå innebörden av dessa. Dessa är kopplade till varandra och när termen \emph{derivata }nämns menar man på \emph{lutningen}. \emph{Derivatan} eller \emph{lutningen} tyder på förändringshastigheten i en viss tidpunkt. Den kända formeln för \emph{derivatan}, även kallad \emph{derivatans definition}: 
%
\begin{align*}
  f'(x)=\frac{f(a+h)-f(a)}{h}
\end{align*}
%
Fomeln menar på att \emph{lutningen} för en funktion, bestäms genom att låta 


\newpage
\section{En backes lutning}
\label{sec:uppg1}


Uppgiftens lösande sker på följande sätt:
\begin{quote} 
  Bestäm backens lutning för $x = 0.8$.
\end{quote}
%
\emph{Lösning:} Derivera $y=0.5e^{-x^2}$ för att få ut funktionen för lutningen, funktionen för derivatan ser ut: $y'=xe^{-x^2}$
%
Därefter sätter vi in värdet $x=0.8$ i funktionen för lutningen
%
\begin{align*}
	y'(0.8)=0.8e^{-0.8^2}
\end{align*}
Insättningen av $0.8$ ger lutningen värdet $\approx(-0.42)$.Vilket är svaret på deluppgiften. 
%
%
%
% Den här raderna består helt av kommentarer som 
% LaTeX inte bryr sig om. De betraktas inte heller 
% som tomrader, vilka skulle signalera att texten övergår 
% i ett nytt stycke, för de innehåller ju (i alla fall) ett 
% procenttecken på varje rad. På så sätt kan man få in 
%"blankrader" i sitt manus utan att de ska markera 
% nytt stycke. 
%
\section{Backens brantaste punkt}
\label{sec:uppg2}


Följduppgiften handlar om att leta efter det värde på $x$ i en punkt där backen är som brantast.
\section{Och ännu nästa problem...}
\label{sec:uppgN}


\section{Diskussion [och slutsatser]}
\label{sec:disk}


Sammanfatta vad som avhandlats i dokumentet och sätt det i
sitt sammanhang.
%
\begin{thebibliography}{99}
%
\bibitem{latexcompanion} 
Michel Goossens, Frank Mittelbach, and Alexander Samarin. 
\textit{The \LaTeX\ Companion}. 
Addison-Wesley, Reading, Massachusetts, 1993.
%
\bibitem{einstein} 
Albert Einstein. 
\textit{Zur Elektrodynamik bewegter K{\"o}rper}. (German) 
[\textit{On the electrodynamics of moving bodies}]. 
Annalen der Physik, 322(10):891–921, 1905.
%
\end{thebibliography}
%
\end{document}
