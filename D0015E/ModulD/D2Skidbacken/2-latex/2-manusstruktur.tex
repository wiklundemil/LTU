\documentclass[a4paper,12pt]{article}
\usepackage[swedish]{babel}
\usepackage[utf8]{inputenc}
\usepackage{amsmath, amsthm, amssymb}
\usepackage[a4paper,includeheadfoot,margin=2.54cm]{geometry}
\usepackage[left]{lineno}
\newtheorem{theorem}{Theorem}

\newcommand*\patchAmsMathEnvironmentForLineno[1]{%
  \expandafter\let\csname old#1\expandafter\endcsname\csname #1\endcsname
  \expandafter\let\csname oldend#1\expandafter\endcsname\csname end#1\endcsname
  \renewenvironment{#1}%
     {\linenomath\csname old#1\endcsname}%
     {\csname oldend#1\endcsname\endlinenomath}}% 
\newcommand*\patchBothAmsMathEnvironmentsForLineno[1]{%
  \patchAmsMathEnvironmentForLineno{#1}%
  \patchAmsMathEnvironmentForLineno{#1*}}%
\AtBeginDocument{%
\patchBothAmsMathEnvironmentsForLineno{equation}%
\patchBothAmsMathEnvironmentsForLineno{align}%
\patchBothAmsMathEnvironmentsForLineno{flalign}%
\patchBothAmsMathEnvironmentsForLineno{alignat}%
\patchBothAmsMathEnvironmentsForLineno{gather}%
\patchBothAmsMathEnvironmentsForLineno{multline}%
}
\renewcommand\linenumberfont{\normalfont\bfseries\small}


\title{Dokumenttitel}
%
\author{För Fattare\thanks{email: \texttt{forfat-X@student.ltu.se}} \\  
        ~ \\
        Luleå tekniska universitet \\ 
        971 87 Luleå, Sverige}
%          
\date{\today}


\begin{document}


\linenumbers


\maketitle


\begin{abstract}
  \LaTeX-manus ska vara lätta att läsa och förstå. Detta uppnås genom
  logisk gruppering av manusets delar som gör manuset
  strukturerat. 
\end{abstract}


\section{Introduktion}


Eftersom \LaTeX-manus ska läsas och förstås av inte bara maskiner utan
också männsikor ska de vara strukturerade och välskrivna. Vi säger att
innehållet ska vara \emph{logiskt grupperat}. Med detta menas att
kommandon och andra grupperingsinstruktioner skrivs på ett sätt så det
lätt går att se vilka delar som hör ihop och vad som påverkas/är del
av/innesluts av vad. Manus skrivna på detta sätt är inte bara en fröjd
att läsa, utan minskar mängden kryptiska fel och misstag man som
författare gör under skrivarbetet. 


Verktygen för att uppnå en tydlig och ändamålsenlig logisk gruppering
är indentering, tomrader (som skiljer logiska avsnitt åt) och väl
avvägd radbrytning. Radlängden är också medvetet anpassad så fyllda
rader är ungefär lika långa och inte är längre än att de är bekväma
att läsa. Med indentering menas att mellanslag läggs till i början av
rader så att radens innehåll förskjuts åt höger. En väl avvägd
radbrytning innebär en brytning av en rad på ett oväntat ställe men på
så sätt att den logiska grupperingen ändå tydligt framgår. 


Manuset till detta dokument visar hur ett manus kan skrivas för att
vara lätt att läsa och förstå. När du läser manuset till detta
dokument lägg således särskilt märke till hur indentering och
radbrytning används för att hålla samma och tydliggöra logiska
avsnitt. Notera också särskilt den aktiva och högst medvetna
användningen av procenttecken i manuset, för att hålla samma logiska
avsnitt och ändå skapa tomrum i vertikalled. Alltså, var, när du själv
skriver \LaTeX-manus, alltid noga med
%
\begin{itemize}
%
  \item indentering,
%
  \item tomrader,
%
  \item radlängder,
%
  \item radbrytningar och
%
  \item placeringen av procenttecken för att hålla samman. 
%
  \end{itemize}
%
Det mesta av själva sakinnehållet i detta dokument -- rubriker, texter 
och formler -- är dock nonsens och ska ignoreras.


\section{Lorem ipsum}
\label{sec:lorem}


Lorem ipsum dolor sit amet, consectetur adipiscing elit. Sed vitae
ultricies tortor. Cras dapibus dui id volutpat
porta~\ref{sec:suspendisse} et~\ref{sec:curabitur}. 


Mauris scelerisque orci at metus maximus semper vitae nec
est. Curabitur $A$ dapibus, erat non congue bibendum, tellus neque
efficitur $n$ et $m$ velit, a vehicula erat est nec nibh. Phasellus
$x$ aliquam ipsum in placerat tristique. 


\subsection{Phasellus elementum}
\label{subsec:phasellus}


Cras sed massa justo. Phasellus elementum tortor et turpis varius, et
vestibulum tortor tristique. Ut et erat quis nisi fringilla commodo
nec a lectus. Etiam accumsan lorem non mi gravida auctor. Aenean vel
sagittis ante, quis commodo augue, 
%
\begin{equation}
  \label{eq:1}
  A_{nm}(x) = \sum_{i = n}^m x^{\frac{1}{i}},
\end{equation}
%
mauris magna ligula, condimentum ut felis ac, egestas malesuada
velit. In vitae varius ipsum.


Fusce vel eleifend arcu. Proin diam turpis equation~\ref{eq:1},
lacinia in ex fermentum, tristique consectetur velit. Phasellus
ultricies vel lectus sit amet fermentum. 
%
\begin{enumerate}
%  
  \item Orci varius natoque penatibus et magnis dis parturient montes,
        nascetur ridiculus mus.
%
  \item Pellentesque iaculis augue in efficitur ultricies.
%
  \item Pellentesque habitant morbi tristique senectus et netus et
    malesuada fames ac turpis egestas.
%
\end{enumerate}
%
Vivamus ultricies vestibulum sagittis. Cras sed tempus ante. Integer
fermentum, nisl sit amet convallis feugiat, neque augue feugiat magna,
non mattis velit eros in augue.  


\subsection{Aenean dignissim}
\label{subsec:aenean}


Morbi sagittis ac elit vel fringilla. Nunc luctus, quam et ornare
congue, tortor libero dignissim lectus, sit amet placerat nunc orci et
est:
%
\begin{center}
  \textbf{Aenean dignissim tincidunt maximus.}
\end{center}
%
Sed malesuada leo sed vestibulum elementum.


Interdum et malesuada fames ac ante ipsum primis in
faucibus\footnote{Curabitur mattis turpis eget lectus rutrum 
auctor.}. Aliquam eget commodo nulla, 
%
\begin{equation}
  \label{eq:2}
  \sin^2 \alpha + \cos^2 \alpha = 1,
\end{equation}
%
aliquet porttitor nisi. Proin fermentum enim vel arcu suscipit
posuere. Sed libero elit, dignissim id purus hendrerit, bibendum
finibus magna. 


\section{Suspendisse}
\label{sec:suspendisse}


Maecenas quam lorem, accumsan sit amet nisi in, pharetra hendrerit
tellus. Integer sit amet quam accumsan, blandit dui nec, finibus
erat. Fusce ut ultricies massa. Mauris hendrerit dapibus enim ac
lacinia.
%
\begin{table}[h]
%
  \begin{center}
%
    \begin{tabular}{|l|c|c|c|}
      \hline
      Suspendisse  & $M$ & $N$ & $P_M/P_N$ \\
      \hline
      Etiam        & XII & MXM & II        \\
      \hline
      Pellentesque & XVI & I   & MMLC      \\
      \hline
      Cras         & DIV & CCC & CLI       \\
      \hline
    \end{tabular}
%
    \caption{\label{tab:1} Suspendisse varius ut felis at pharetra.} 
%
  \end{center}
%
\end{table}
%
Nunc sed tellus augue.


Etiam diam massa, mollis convallis malesuada ac, suscipit eget
neque. Donec congue, mi a pellentesque porttitor, libero urna mattis
erat, 
%
\begin{displaymath}
  \frac{\partial u}{\partial t}
    = h^2 \left(
            \frac{\partial^2 u}{\partial x^2} +
            \frac{\partial^2 u}{\partial y^2} +
            \frac{\partial^2 u}{\partial z^2}
          \right), 
\end{displaymath}
%
rutrum porttitor dui lectus a sapien. Sed condimentum diam at
euismod laoreet.


\section{Curabitur}
\label{sec:curabitur}


Quisque egestas arcu non massa placerat suscipit. Curabitur ultrices,
leo posuere dapibus placerat, justo nibh~\pageref{sec:lorem} gravida
risus, eu dignissim felis tellus non magna equation~\ref{eq:1} et~\ref{eq:2}.
Cras ipsum mauris,
%
\begin{displaymath}
  e^{i\pi} + 1 = 0,
\end{displaymath}
%
semper vitae leo vitae, mattis porttitor mauris. 


Ut pellentesque gravida cursus. Aenean sit amet ex sollicitudin,
scelerisque ante a, ultrices metus.
%
\begin{theorem}
%
  Pellentesque convallis urna a viverra placerat,
%
  \begin{equation}
    \label{eq:4}
%
    H(m, n) = 
    \begin{cases}
      m    & \text{if } n < 0 \land m > 0 \\
      n,   & \text{if } n > 0 \land m < 0 \\
      n+m, & \text{otherwise}.
    \end{cases}
%
  \end{equation}
%
\end{theorem}


Proin eu ullamcorper dolor. In pharetra eros eget leo lacinia, quis
bibendum libero scelerisque. Limit $\lim_{x\to\infty} f(x)$, fusce
vestibulum blandit volutpat,
%
\begin{displaymath}
  \lim_{x \to \infty} f(x). 
\end{displaymath}
%
Duis et massa ac elit vestibulum rhoncus a sed nibh. Proin pretium
nibh et est placerat, 
%
\begin{equation}
  \label{eq:split}
  %
  \begin{split}
    \tau &= \frac{\pi r^2}{2}   \\
         &= \frac{1}{2} \pi r^2,
  \end{split}
  %     
\end{equation}
%
in vulputate lacus consectetur,
%
\begin{align*}
  (x + h)^2 - x^2 &= x^2  + 2xh + h^2 - x^2 && \text{(pulvinar nulla)} \\
                  &= 2xh  + h^2             && \text{(luctus magna)} \\
                  &= h(2x + h).
\end{align*}


In condimentum sem vitae faucibus luctus. Aenean quis augue elementum,
varius nisi ut, scelerisque augue. Aenean non pulvinar nulla. Duis id
laoreet tellus, a egestas felis. Morbi at leo lobortis, luctus magna
et, dapibus nisi. 


In nec finibus nisl. Donec malesuada pellentesque aliquet. Quisque et
arcu lobortis, tristique metus a, fermentum ex. Sed ex urna, pretium a
erat eu, elementum pretium justo~\cite{latexcompanion,einstein}. Maecenas
tincidunt massa id est hendrerit, id interdum enim feugiat:
%
\begin{displaymath}
  E = mc^2.
\end{displaymath}
%
\begin{thebibliography}{99}
%
\bibitem{latexcompanion} 
Michel Goossens, Frank Mittelbach, and Alexander Samarin. 
\textit{The \LaTeX\ Companion}. 
Addison-Wesley, Reading, Massachusetts, 1993.
%
\bibitem{einstein} 
Albert Einstein. 
\textit{Zur Elektrodynamik bewegter K{\"o}rper}. (German) 
[\textit{On the electrodynamics of moving bodies}]. 
Annalen der Physik, 322(10):891–921, 1905.
%
\end{thebibliography}
%
\end{document}