\documentclass[a4paper,12pt]{article}
\usepackage[swedish]{babel}
\usepackage[utf8]{inputenc}
\usepackage{amsmath, amsthm, amssymb}
\usepackage[a4paper,includeheadfoot,margin=2.54cm]{geometry}
\usepackage[left]{lineno}
\usepackage{mdframed}
\usepackage{listings}
%\usepackage{xcolor}
\lstset{
  basicstyle=\ttfamily,
  columns=fullflexible,
  keepspaces=true,
  escapeinside=||,
  frame = single
}
\lstset{literate=
  {á}{{\'a}}1 {é}{{\'e}}1 {í}{{\'i}}1 {ó}{{\'o}}1 {ú}{{\'u}}1
  {Á}{{\'A}}1 {É}{{\'E}}1 {Í}{{\'I}}1 {Ó}{{\'O}}1 {Ú}{{\'U}}1
  {à}{{\`a}}1 {è}{{\`e}}1 {ì}{{\`i}}1 {ò}{{\`o}}1 {ù}{{\`u}}1
  {À}{{\`A}}1 {È}{{\'E}}1 {Ì}{{\`I}}1 {Ò}{{\`O}}1 {Ù}{{\`U}}1
  {ä}{{\"a}}1 {ë}{{\"e}}1 {ï}{{\"i}}1 {ö}{{\"o}}1 {ü}{{\"u}}1
  {Ä}{{\"A}}1 {Ë}{{\"E}}1 {Ï}{{\"I}}1 {Ö}{{\"O}}1 {Ü}{{\"U}}1
  {â}{{\^a}}1 {ê}{{\^e}}1 {î}{{\^i}}1 {ô}{{\^o}}1 {û}{{\^u}}1
  {Â}{{\^A}}1 {Ê}{{\^E}}1 {Î}{{\^I}}1 {Ô}{{\^O}}1 {Û}{{\^U}}1
  {œ}{{\oe}}1 {Œ}{{\OE}}1 {æ}{{\ae}}1 {Æ}{{\AE}}1 {ß}{{\ss}}1
  {ű}{{\H{u}}}1 {Ű}{{\H{U}}}1 {ő}{{\H{o}}}1 {Ő}{{\H{O}}}1
  {ç}{{\c c}}1 {Ç}{{\c C}}1 {ø}{{\o}}1 {å}{{\r a}}1 {Å}{{\r A}}1
  {€}{{\euro}}1 {£}{{\pounds}}1 {«}{{\guillemotleft}}1
  {»}{{\guillemotright}}1 {ñ}{{\~n}}1 {Ñ}{{\~N}}1 {¿}{{?`}}1
}


\title{Förbättring av \LaTeX-manus läsbarhet}
%
\author{Håkan Jonsson\thanks{email: \texttt{hj@ltu.se}. Detta är
      kursmaterial i kursen D0015E Datateknik och ingenjörsvetenskap.} \\  
        ~ \\
        Luleå tekniska universitet \\ 
        971 87 Luleå, Sverige}
%          
\date{\today}


\begin{document}


\maketitle


\begin{abstract}
  \LaTeX-manus ska vara lätta att läsa och förstå. Tre sätt att som
  författare uppnå detta på är a) indentering, b) radbrytning och c)
  justering av raderna i vertikal- och horisontalled.

  
  I detta dokument ger vi några exempel på användning av
  dessa sätt, som hjälp då \LaTeX\ ska användas för att skriva
  rapporter och reflektioner. 
\end{abstract}


\section{Inledning}


Denna rapport går igenom olika sätt att förbättra läsbarheten hos ett
\LaTeX-manus. De primära sättet är indentering, radbrytning och
justering i både vertikal- som horisontalled.


\section{Konsekvent indentering}


Kod ska vara konsekvent indenterad. Med indentering menas att kod
förskjuts in på raden så det klarare framgår vad som logiskt sitter
ihop. Det typiska är att det som finns inne i en omgivning indenteras
i förhållande till de \texttt{begin}- och \texttt{end}-deklarationer
som startar och avslutar omgivningen. Istället för att skriva
%
\begin{verbatim}
\begin{align*} f(x) &= x^4 + x^2 + 1 \\ g(x) &= x^5 + x^3 + x \end{align*}
\end{verbatim}
%
på en enda rad eller
%
\begin{verbatim}
\begin{align*}
f(x) &= x^4 + x^2 + 1 \\
g(x) &= x^5 + x^3 + x
\end{align*}
\end{verbatim}
%
på 4 rader som alla börjar ytterst till vänster så skriver vi
%
\begin{verbatim}
\begin{align*}
        f(x) &= x^4 + x^2 + 1 \\
        g(x) &= x^5 + x^3 + x
\end{align*}
\end{verbatim}
%
dvs visar tydligt att de två funktionerna $f$ och $g$ tillhör
\texttt{align*}-omgivningen genom att indentera dem 8
steg/mellanslag. 

Storleken på indenteringen, antalet steg, ska vara densamma i hela
manuset och även då indentering görs av redan indenterat (omgivningar
i omgivningar). Vanliga storlekar är 2, 4 och 8 (som ovan) men det går att
använda vilken som helst bara man är konsekvent. En större
indentering kan ge bättre läsbarhet men förbrukar också mer av
utrymmet på raden. Och om det finns mycket goda skäl kan indenteringen
undantagsvis skilja i storlek på isolerade ställen i manuset, utan att
för den skull bryta huvudregeln. Här är
ett annat exempel på indentering, där storleken också
är 8 steg.
%
\begin{lstlisting}
Från förutsättningarna gäller att
\begin{displaymath}
        g(x) = \sin x + \cos x. 
\end{displaymath}
Eftersom $a(x) = g(h(x))$, och 
\begin{displaymath}
        h(x) = x^3,
\end{displaymath}
blir således 
\begin{displaymath}
        a(x) = \sin x^3 + \cos x^3.
\end{displaymath}
vilket stämmer med slutsatserna i förra kapitlet.
\end{lstlisting}
%
Det viktiga med indenteringen är att den ska klart och tydligt visa
vad som ingår i vad. Jämför med hur det blir utan indentering. 
%
\begin{lstlisting}
Från förutsättningarna gäller att
\begin{displaymath}
g(x) = \sin x + \cos x. 
\end{displaymath}
Eftersom $a(x) = g(h(x))$, och 
\begin{displaymath}
h(x) = x^3,
\end{displaymath}
blir således 
\begin{displaymath}
a(x) = \sin x^3 + \cos x^3,
\end{displaymath}
vilket stämmer med slutsatserna i förra kapitlet.
\end{lstlisting}
%
Här börjar raderna flyta samman och det är inte lika lätt att se
skillnaden mellan text, formler och kommandon/omgivningar. 


\section{Radbrytning och indentering}


Den indenterade koden från förra exemplet kan ytterligare förbättras
genom radbrytning av formler och text så viktiga delar skiljs ut och
visas upp.  
%
\begin{lstlisting}
Från förutsättningarna gäller att
\begin{displaymath}
        g(x) = \sin x + 
               \cos x. 
\end{displaymath}
Eftersom 
        $a(x) = g(h(x))$, 
och 
\begin{displaymath}
        h(x) = x^3,
\end{displaymath}
blir således 
\begin{displaymath}
        a(x) = \sin x^3 + 
               \cos x^3.
\end{displaymath}
vilket stämmer med slutsatserna i förra kapitlet.
\end{lstlisting}
%
Nu framgår allt ännu tydligare. Indenteringen är inte helt konsekvent,
för raderna med \verb|\cos| är mer indenterade än övriga indenterade
rader, men ändamålsenlig eftersom den större indenteringen justerar
termerna så man ser att de sitter ihop. Eftersom det finns ett sådant
starkt vägande skäl är inkonsekvensen i indentering OK. 


Obs! Hur koden skrivs i manuset behöver inte -- ska inte! -- slaviskt följa
hur det typsatta resultatet blir. Titta t ex på hur 
    \verb|$a(x) = g(h(x))$| (på rad 7 i exemplet) 
satts på en egen indenterad rad, för att visas upp \emph{i manuset}. 
Sen kommer det i alla fall att typsättas \emph{löpande mitt i texten}. 


\section{Justering med tomrader}


Tomrader separerar visuellt i vertikalled, vilket också kan öka
läsbarheten. För att få in en tomrad i koden där såna egentligen inte
går att ha med kan vi använda rader som endast har ett procenttecken
först på raden. Vi exemplifierar detta genom att lägga in såna rader i
exempelkoden från föregående avsnitt.
%
\begin{lstlisting}
Från förutsättningarna gäller att
%
\begin{displaymath}
        g(x) = \sin x + 
               \cos x. 
\end{displaymath}
%
Eftersom 
        $a(x) = g(h(x))$, 
och 
%
\begin{displaymath}
        h(x) = x^3,
\end{displaymath}
%
blir således 
%
\begin{displaymath}
        a(x) = \sin x^3 + 
               \cos x^3.
\end{displaymath}
%
vilket stämmer med slutsatserna i förra kapitlet.
\end{lstlisting}
%
Detta glesar ut vertikalt, så framförallt alla omgivningar syns tydligare.


Kanske kan man lockas använda helt tomma rader för att separera i
vertikalled? Tomrad, såväl en som godtyckligt många efter varandra,
är dock ett ''kommando'' som betyder nytt stycke i \LaTeX. Tomrad kan
därför inte användas istället för procentrad.  


För att vara mycket tydliga var i texten nya stycken börjar använder
vi alltid två (2) timrader efter varandra. Detsamma gör vi före och
efter avsnittsrubriker. Om det ska vara nytt stycke i texten före
meningen som inleds med \texttt{Eftersom} skriver vi
%
\begin{lstlisting}
...
               \cos x. 
\end{displaymath}


Eftersom 
        $a(x) = g(h(x))$, 
och 
%
\begin{displaymath}
...
\end{lstlisting}
%
Så länge vi är någorlunda konsekventa och förbättrar läsbarheten kan
vi variera relativt fritt hur vi skriver vårt manus med vertikala
justeringar och mellanrum. 


\subsection{Radbrytning och separering}


Här utgår vi nu från koden
%
\begin{lstlisting}
\begin{displaymath}
  f(x)=x^3+2x^2+x+x+2x^2+x^3=2x^3+2x^2+4x=2x(x^2+2x+1)=2x(x+1)^2.
\end{displaymath}
\end{lstlisting}
%
som ger
\begin{displaymath}
  f(x)=x^3+2x^2+x+x+2x^2+x^3=2x^3+4x^2+2x=2x(x^2+2x+1)=2x(x+1)^2.
\end{displaymath}
%
Eller gör den? Koden består av flera likadana delar och är lite
svårläst, eller hur,  så låt oss se hur vi med enkla medel kan öka
dess läsbarhet.  


I matematikomgivningar som \texttt{displaymath}, \texttt{equation}
och \texttt{align} spelar, precis som i vanlig löptext, radbrytning
ingen roll för \LaTeX. Det är ordningen mellan manusets delar som de
kommer när vi läser manuset uppifrån och ner, och varje rad från
vänster till höger, som har betydelse. Vi kan därför bryta raderna så
koden ändras till 
%
\begin{lstlisting}
\begin{displaymath}
  f(x)=x^3+2x^2+x+x+2x^2+x^3
      =2x^3+4x^2+2x
      =2x(x^2+2x+1)
      =2x(x+1)^2.
\end{displaymath}
\end{lstlisting}
%
och fortfarande få exakt samma utseende på det typsatta resultatet. De
kortare (separata) manusraderna är enklare att läsa än när de alla
bildade en enda lång rad. Med separerande mellanslag inne i raderna kan vi
förbättra läsbarheten ytterligare.  
%
\begin{lstlisting}
\begin{displaymath}
  f(x) = x^3  + 2x^2 + x  + x + 2x^2 + x^3
       = 2x^3 + 4x^2 + 2x
       = 2x(x^2 + 2x + 1)
       = 2x(x + 1)^2.
\end{displaymath}
\end{lstlisting}
%
Här är termerna och plustecknen på de två första formelraderna justerade i
horisontalled så de ska hamna i linje vertikalt. Även efterföljande
formelrader är utglesade för ökad läsbarhet. Att detta fungerar
beror på att \LaTeX-systemet (i princip) inte heller bryr sig om extra
mellanslag på en rad. Även denna kod ger samma typsatta rad som resultat. 


Notera att det till vänster om det tredje plustecknet från slutet av
första raden finns ett extra, andra, mellanslag som om det skulle finnas ett
motsvarande plustecken på raden under att justera mot. Anledningen är
ett försök att visuellt ytterligare separera ut detta plustecken från
termen \texttt{2x} på andra raden. (Och \emph{om} det hade funnits ett
plustecken också på andra raden och efter \texttt{2x} så hade vi
förstås justerat plustecknet mot det.) Naturligtvis skulle vi kunna
justera alla fyra raders innehåll mot varandra. 
%
\begin{lstlisting}
\begin{displaymath}
  f(x) = x^3    + 2x^2 + x  + x + 2x^2 + x^3
       = 2x^3   + 4x^2 + 2x
       = 2x(x^2 + 2x   + 1)
       = 2x(x   + 1)^2.
\end{displaymath}
\end{lstlisting}
%
Återigen påverkar detta inte typsättningen men väl läsbarheten.


Om extra mellanslag som ovan är lämpliga eller ej är svårt att säga
generellt. Men det ökar i regel läsbarheten att justera termer och
annat mot varandra, och att skapa tomma (visuella) 
kolumner i långa uttryck som sträcker sig över flera rader. Utan det
extra mellanslaget får vi istället 
%
\begin{lstlisting}
\begin{displaymath}
  f(x) = x^3  + 2x^2 + x + x + 2x^2 + x^3
       = 2x^3 + 4x^2 + 2x
       ...
\end{displaymath}
\end{lstlisting}
%
vilket också går bra, även om den tomma kolumnen försvann. 


Ovan är termerna vänsterjusterade. Vi kan också välja att justera dem
åt höger. 
\begin{lstlisting}
\begin{displaymath}
  f(x) =  x^3 + 2x^2 +  x + x + 2x^2 + x^3
       = 2x^3 + 4x^2 + 2x
       ...
\end{displaymath}
\end{lstlisting}
%
Skillnaden syns t ex på första raden, där termerna är fortsatt läsbara
(eller kanske t~o~m snäppet lättare att läsa -- en smaksak). Vi har nu
också fått tillbaka den försvunna kolumnen.


Att som ovan stuva om i kod för att öka dess läsbarhet kan också ge
idéer till hur det typsatta resultatets läsbarhet kan förbättras. I
fallet ovan vore ett alternativ att dela upp vid likamedtecknen och
rada upp delarna med \texttt{align*}. 
%
\begin{lstlisting}
\begin{align*}
  f(x) &= x^3   + 2x^2  + x   +  x  +  2x^2  +  x^3  \\
       &= 2x^3  + 4x^2  + 2x                   \\
       &= 2x (x^2 + 2x  + 1)                   \\
       &= 2x (x + 1)^2.
\end{align*}
\end{lstlisting}
%
Som synes måste inte alla rader vara justerade mot varandra; olikheter
måste få finnas kvar (och slutraderna är på annan form än de två
första som dock går bra att justera mot varandra).

Tanken var inte att inkludera en flerradsekvation i texten. Men om vi
jämför blir resultatet tydligare med flerradsekvation än utan, och som
väntat betydligt lättare att läsa och förstå.
%
\begin{align*}
  f(x) &= x^3  + 2x^2 + x + x + 2x^2 + x^3 \\
       &= 2x^3 + 4x^2 + 2x \\
       &= 2x(x^2 + 2x + 1) \\
       &= 2x(x + 1)^2. 
\end{align*}
%
För jämförelsens skull upprepar vi hur det första typsatta resultatet
ser ut. 
\begin{displaymath}
  f(x)=x^3+2x^2+x+x+2x^2+x^3
  =2x^3+4x^2+2x=2x(x^2+2x+1)=2x(x+1)^2.
\end{displaymath}
%
Radbrytning och horisontell justering kan alltså inte bara öka ett
\LaTeX-manus läsbarhet utan även ge idéer om hur läsbarheten hos det
färdiga resultatet kan förbättras. 


Hur var det förresten nu med koden i början av hela detta
exempel som påstods ge den typsatta raden i början..? Var den
korrekt..? Nej, den innehåller ett medvetet fel(!) Den som noga
granskade koden upptäckte en bit in att en 2:a och en 4:a var
omkastade jämfört med hur de förekommer i det typsatta
resultatet. Undvik sån svårläst kod. 


\section{Radbrytning i vanlig text}


Vi ska nu se hur läsbarheten hos kod som beskriver ett litet
textstycke kan förbättras med indentering, radbrytning och justering 
vertikalt och horisontellt. Vi utgår från följande kod.  
%
\begin{lstlisting}
Vi ska nu se hur läsbarheten hos kod som beskriver ett litet
textstycke kan förbättras med indentering, radbrytning och justering 
vertikalt och horisontellt. Vi utgår från följande kod. 
\end{lstlisting}
%
Först ser vi att det finns en uppräkning i texten. Även om den inte
ska typsättas som en lista hindrar inget att vi, i manuset, skriver
den som en lista. 
%
\begin{lstlisting}
Vi ska nu se hur läsbarheten hos kod som beskriver ett litet
textstycke kan förbättras med 
    indentering, 
    radbrytning och 
    justering vertikalt och horisontellt. 
Vi utgår från följande kod. 
\end{lstlisting}
%
Detta lyfter fram de tre uppräknade sätten. Däremot påverkar det inte
typsättningen. Och vill vi vara ännu tydligare i manuset kan vi också
separera ut i vertikalled. 

%
\begin{lstlisting}
Vi ska nu se hur läsbarheten hos kod som beskriver ett litet
textstycke kan förbättras med 
%
    indentering, 
%
    radbrytning och 
%
    justering vertikalt och horisontellt. 
%
Vi utgår från följande kod. 
\end{lstlisting}
%
Det blir ändå precis samma typsatta resultat.


\section{Ett avslutande kombinationsexempel}


Vi ska nu titta närmare på ett blandat exempel. Här är version 1 av
manuset:
%
\label{page:v1}
\begin{lstlisting} 
Med våra värden $a_2=-25.35$, $a_3=155.925$ och $a_4=-250$
blir \begin{align*} p&=155.925-(-25.35)^2/3 \\ &= -58.2825 
\end{align*} och \begin{align*} q&=-291,625+2(-25-35)^3/27-
18(-25.35)\cdot 155.925/27\\ &=-180.761, \end{align*} I uppgiften är
vidare givet att $x=2$. Detta insatt i basekvationen ger oss volymen
$V(2) = 4\cdot 2^3 -101.4\cdot 2^2 + 623.7\cdot 2 = 32 - 405.6 +
1247.4= 873.8 ~\text{cm}^3.$ Detta är ungefär $0.87$~liter. 
\end{lstlisting}
%
Detta är extremt svårläst! Usch! Men \LaTeX-systemet har inga problem med det,
utan läser bara rad för rad och vad som står på raderna.
Egentligen är ovanstående ''gröt'' detsamma som nedanstående,
något mera lättlästa, och indenterade, manus som vi istället utgår
från i resten av avsnittet.  
%
\begin{lstlisting}
Med våra värden $a_2=-25.35$, $a_3=155.925$ och $a_4=-250$ blir
\begin{align*}
    p&=155.925-(-25.35)^2/3 \\
    &= -58.2825
\end{align*}
och
\begin{align*}
    q&=-291,625+2(-25-35)^3/27-18(-25.35)\cdot 155.925/27\\
    &=-180.761,
\end{align*}
I uppgiften är vidare givet att $x=2$. Detta insatt i basekvationen ger
oss volymen $V(2) = 4\cdot 2^3 -101.4\cdot 2^2 + 623.7\cdot 2 = 32 -
405.6 + 1247.4= 873.8 ~\text{cm}^3.$ Detta är ungefär $0.87$~liter.
\end{lstlisting}
%
Här kan man i alla fall enklare förstå vad för text det ska bli. Denna
senare kod ger, liksom ''gröten'', följande typsatta
resultat. (Observera att basekvationen som det refereras till inte är
med i exempelkoden.)   
%
\begin{mdframed}
  Med våra värden $a_2=-25.35$, $a_3=155.925$ och $a_4=-250$ blir
  \begin{align*}
    p&=155.925-(-25.35)^2/3 \\
     &= -58.2825
  \end{align*}
  och
  \begin{align*}
    q&=-291,625+2(-25-35)^3/27-18(-25.35)\cdot 155.925/27\\
     &=-180.761,
  \end{align*}
  I uppgiften är vidare givet att $x=2$. Detta insatt i basekvationen
  ger oss volymen
  $V(2) = 4\cdot 2^3 -101.4\cdot 2^2 + 623.7\cdot 2 = 32 - 405.6 +
  1247.4= 873.8 ~\text{cm}^3.$ Detta är ungefär $0.87$~liter.
\end{mdframed}


Så, hur förbättrar vi manuset? Vi börjar med att radbryta uppräkningen
på första raden och glesa ut.
%
\begin{lstlisting}
Med våra värden 
    $a_2 = -25.35$, 
    $a_3 =  155.925$ och 
    $a_4 = -250$ 
blir
\begin{align*}
...
\end{lstlisting}
%
Sen glesar vi ut matematiken i \texttt{align*}-omgivningarna och
justerar termer. 
%
\begin{lstlisting} 
...
\begin{align*}
    p &=   155.925 - (-25.35)^2 / 3 \\
      &= - 58.2825
\end{align*}
och
\begin{align*}
    q &= -291,625 + 2(-25 - 35)^3/27 - 18(-25.35) \cdot 155.925 / 27 \\
      &= -180.761,
\end{align*}
...
\end{lstlisting}
%
En av kodraderna ovan är lång och lite svårläst, så vi radbryter den
och justerar.
%
\begin{lstlisting} 
...
\begin{align*}
    q &= - 291,625  
         + 2(-25 - 35)^3 / 27 
         - 18(-25.35) \cdot 155.925 / 27 \\
      &= - 180.761,
\end{align*}
...
\end{lstlisting}
%
Kvar är de avslutande tre raderna kod.
%
\begin{lstlisting}
...
I uppgiften är vidare givet att $x=2$. Detta insatt i basekvationen ger
oss volymen $V(2) = 4\cdot 2^3 -101.4\cdot 2^2 + 623.7\cdot 2 = 32 -
405.6 + 1247.4= 873.8 ~\text{cm}^3.$ Detta är ungefär $0.87$~liter.
\end{lstlisting}
Vi delar upp den, radbryter, indenterar, glesar och justerar (och separerar
t~o~m med procentrader).
%
\begin{lstlisting}
...
I uppgiften är vidare givet att 
%
    $x = 2$. 
%
Detta insatt i basekvationen ger oss volymen 
%
    $V(2) = 4     \cdot 2^3 - 
            101.4 \cdot 2^2 +
            623.7 \cdot 2 
%
          = 32 - 
            405.6 + 
            1247.4
%
          = 873.8~\text{cm}^3.$ 
%
Detta är ungefär $0.87$~liter.
\end{lstlisting}
%
In med några procentrader för att tydligare visa omgivningarna, och
sen har vi ändrat hela manussnutten(!)
%
\label{page:vn} 
\begin{lstlisting}
Med våra värden 
    $a_2 = -25.35$, 
    $a_3 =  155.925$ och 
    $a_4 = -250$ 
blir
%
\begin{align*}
    p &=   155.925 - (-25.35)^2 / 3 \\
      &= - 58.2825
\end{align*}
%
och
%
\begin{align*}
    q &= - 291,625  
         + 2(-25 - 35)^3 / 27 
         - 18(-25.35) \cdot 155.925 / 27 \\
      &= - 180.761,
\end{align*}
%
I uppgiften är vidare givet att 
%
    $x = 2$. 
%
Detta insatt i basekvationen ger oss volymen 
%
    $V(2) = 4     \cdot 2^3 - 
            101.4 \cdot 2^2 +
            623.7 \cdot 2 
%
          = 32 - 
            405.6 + 
            1247.4
%
          = 873.8~\text{cm}^3.$ 
%
Detta är ungefär $0.87$~liter.
\end{lstlisting}
%
Det går att på nytt gå igenom alla raderna och ytterligare justera och
separera för ännu bättre läsbarhet, men denna första genomgång räcker
redan långt. Vi typsätter nu denna omformulerade kod. 

\begin{mdframed}
Med våra värden 
    $a_2 = -25.35$, 
    $a_3 =  155.925$ och 
    $a_4 = -250$ 
blir
%
\begin{align*}
    p &=   155.925 - (-25.35)^2 / 3 \\
      &= - 58.2825
\end{align*}
%
och
%
\begin{align*}
    q &= - 291,625  
         + 2(-25 - 35)^3 / 27 
         - 18(-25.35) \cdot 155.925 / 27 \\
      &= - 180.761,
\end{align*}
%
I uppgiften är vidare givet att 
%
    $x = 2$. 
%
Detta insatt i basekvationen ger oss volymen 
%
    $V(2) = 4     \cdot 2^3 - 
            101.4 \cdot 2^2 +
            623.7 \cdot 2 
%
          = 32 - 
            405.6 + 
            1247.4
%
          = 873.8~\text{cm}^3.$ 
%
Detta är ungefär $0.87$~liter.
\end{mdframed}
%
Detta är fortfarande exakt samma typsatta resultat som i början av
detta avsnitt av rapporten. Vi har bara förbättrat kodens
läsbarhet.


\section{Diskussion}

Det finns, som alla exempel i denna rapport visar, många olika sätt att förbättra
ett manus läsbarhet på. Med små medel kan man dramatiskt göra ett
manus mer läbegripligt och lättläst. Jämför t~ex det första (extrema)
manuset (''gröten'') på sidan~\pageref{page:v1} med sista versionen på
sidan~\pageref{page:vn}. 


Det viktiga att ta med sig efter att ha läst detta dokument, och
förutom de sätt som vi kan öka läsbarheten på, är att \emph{du} har en
\emph{skyldighet} att skriva manus så att dina medförfattare tycker att de är
läsbara. Och på samma sätt har var och en av dina medförfattare samma
skyldighet mot dig och övriga. Användningen av verktyg som indentering,
radbrytning samt justering i vertikal- och horisontalled måste vara
sund.  


Ska man samarbeta om att skriva ett \LaTeX-manus, eller ett
datorprogram i största allmänhet, är det därför klokt att komma överens (i
förväg) om hur man ska skriva. Detta brukar kallas
\emph{kodningsstandard} och finns för de flesta riktiga
programmeringsspråk. 
%
\end{document}
